% !TEX root = ../main.tex

\chapter{Introduction}
\label{cp:introduction}

\fncydesc{Copyright notice}{This document and the figures therein are licensed under the Creative Common BY-SA 4.0 Licence.}

This document is suitable for academic reports. You can always replace the logo of Massey University with that of yours. If you can, try to find or create a vectorised graphic for the logo. Vectorised logos of 8 universities in NZ plus Unitec, as well as the method for extracting the logo of other universities are provided here:

\verb|https://www.io.ac.nz/blog/nz-uni-logos-vector/|.

Some useful Latex tips and tricks are given below.

\section{Citation}
You can cite like this:

\begin{itemize}
  \item ``\verb|\cite{Einstein1905}|'' gives you \cite{Einstein1905} for a single citation
  \item ``\verb|\cite{Goossens1993,Knuth2000}|'' gives you \cite{Goossens1993,Knuth2000} for multiple citation
  \item ``\verb|\cite[Pg. 42]{Einstein1905}|'' gives you \cite[Pg. 42]{Einstein1905} for extra information such as page number (quite useful if you are citing a whole book or a long paper)
\end{itemize}

\section{Equation}

\subsection{One equation without number}
This is an equation without numbering. You cannot reference it.
\begin{equation*}
a = b
\end{equation*}

\subsection{One equation with number}
Equation \ref{eq:eq1} is the same equation with numbering, so you can reference it as usual.
\begin{equation}
a = b
\label{eq:eq1}
\end{equation}

Eq \ref{eq:eq3} is a case equation.
\begin{equation}
f(n) = 
\begin{dcases}
a & \text{if } \quad 0 < n\\
b & \text{if } \quad n = 0 \\
c & \text{if } \quad 0 < n \le p \\
d & \text{if } \quad n > p
\end{dcases}
\label{eq:eq3}
\end{equation}
%
\subsection{Multiple equations without number}
This is a set of equation, aligned by the equal sign (=). No reference
\begin{equation*}
\begin{aligned}
a &= b \\
c &= d
\end{aligned}
\end{equation*}
%
\subsection{Multiple equations with one number}
\ref{eq:eq2} is the same set of equation with one single equation number for the whole set.
\begin{equation}
\begin{aligned}
a &= b \\
c &= d
\end{aligned}
\label{eq:eq2}
\end{equation}

\subsection{Multiple equations with one number each equation}
\ref{eq:eq4} is a set of equations, each assigned an equation number. You can reference them directly like this: \ref{eq:eq4a} and \ref{eq:eq4b}
\begin{subequations}
  \begin{align}
  a &= b \label{eq:eq4a} \\
  c &= d \label{eq:eq4b}
  \end{align}
  \label{eq:eq4}
\end{subequations}

\subsection{Remove number from one equation in the set}
If you want to omit the number for any particular equation, end it with \verb|\nonumber|. Now only Eq. \ref{eq:eq6} has a number while the one before it does not.
\begin{align}
a &= b \nonumber\\
c &= d \label{eq:eq6}
\end{align}

\section{Useful shortcuts}
\subsection{Todo list}
This is how you insert a to do item:
\tudu{To do: convert the code in Code Block \ref{code:helloword-in-c} to Java}

\subsection{Missing figure}
Figure \ref{fig:missing-figure} is a missing figure.
\begin{figure}[!h]
  \misfig{Missing figure}
  \caption[Missing figure 1]{This is to remind you to insert an actual figure here before submitting this report}
  \label{fig:missing-figure}  
\end{figure}

\subsection{Direct quote}
And here is a direct quote:
\begin{fancyquote}
  There are many people who feel that it is useless and futile to continue talking about peace and non-violence against a government whose only reply is savage attacks on an unarmed and defenseless people.
  
  \raggedleft\textit{Nelson Mandela}
\end{fancyquote}

\subsection{Definition}
A definition:
\fncydesc{Integration}{The process of finding a function, given its derivative, is called anti-differentiation (or integration). If $F'(x) = f(x)$, we say $F(x)$ is an anti-derivative of $f(x)$.}


\section{Figures}
\subsection{Code block}
Code block \ref{code:helloword-in-c} shows the Hello World example written in C. You can set different colour scheme for comments, keywords and strings in setup.tex.

\lstset{language=C,caption={Hello world in C},label=code:helloword-in-c}
\begin{lstlisting}
#include <stdio.h>
#define N 10
/* Block
* comment */

int main() {
  int i;
  
  // Line comment.
  puts("Hello world!");
  
  for (i = 0; i < N; i++) {
    puts("LaTeX is also great for programmers!");
  }
  
  return 0;
}
\end{lstlisting}

\subsection{A graph}
To achieve best quality, you should always insert vectorised figures. This can be achieved with figures exported from Matlab or hand-drawn in Adobe Illustrator, Inkscape, or Corel Draw. A detail tutorial is available at \verb|https://www.io.ac.nz/blog/matlab-to-latex/|. For an example, have a look at Fig. \ref{fig:elc}. The whole figure is plotted in Matlab, then Inkscape separate the graph part (lines and curves...) to a \verb|pdf| file and the text part to a \verb|pdf_tex| file, so that the text is actually rendered by Latex to maintain consistency in the document.


\begin{figure}[!h]
  \def\svgwidth{\linewidth}
  \small
  \input{figures/equal-loudness-contour.pdf_tex}
  \caption[Equal-loudness-contour]{Equal loudness contour according to ISO 226:2003}
  \label{fig:elc}
\end{figure}


\subsection{Two graphs, side by side}
Figure \ref{fig:fig3} shows two subfigure side by side, The total width span one whole line, with the first figure occupies 47.5\% total width and pushed all the way to the left, the second figure occupies 47.5\% total width and pushed all the way to the right, the space between them is 5\% total width. You can cite the subfigures independent from the whole figure, like this: \verb|\ref{fig:fig3a}| (to get \ref{fig:fig3a}), or just the subfigure's part of the reference number, like this: \verb|\subref{fig:fig3a}| (to get \subref{fig:fig3a})

\noindent
\begin{figure}[!h]
  \begin{subfigure}[b]{.5\linewidth}
    \raggedright\footnotesize
    \def\svgwidth{.95\linewidth}
    \input{figures/dtw_euclidean.pdf_tex}
    \caption{Distance calculated by a $\ell_p$-norm is performed on a linear element-wise basis.}
    \label{fig:fig3a}
  \end{subfigure}%
  \begin{subfigure}[b]{.5\linewidth}
    \raggedleft\footnotesize
    \def\svgwidth{.95\linewidth}
    \input{figures/dtw_dtw.pdf_tex}
    \caption{Distance calculated by a DTW is performed on a warped path.}
    \label{fig:fig3b}
  \end{subfigure}
  \caption[Illustration of $\ell_p$-norm distance and DTW distance]{The alignment of elements for measuring distance between two sequences A (top blue line) and B (bottom black line) by \subref{fig:fig3a}) $\ell_p$-norm distances and \subref{fig:fig3b}) DTW. Red dots are where the warping occurs.}
  \label{fig:fig3}
\end{figure}

\lipsum[3]

\subsection{Figure wrapped by text}
\begin{wrapfigure}{r}{.5\textwidth}
  \vspace{-20pt}
  \raggedright
  \footnotesize
  \def\svgwidth{\linewidth}
  \input{figures/sequences-with-no-complete-match.pdf_tex}
  \caption{Matching sequences that have non-correspondent segment.}
  \label{fig:fig4}
  \vspace{-10pt}
\end{wrapfigure}
%
\lipsum[2]

